\documentclass[12pt]{article}
\usepackage[margin=0.75in]{geometry}
\usepackage{amsmath, amssymb}

\begin{document}

\section*{Inner Product Spaces}

For a vector space \( V \), an inner product \(\langle \cdot, \cdot \rangle\) satisfies the following four postulates (applicable to real vector spaces):

\begin{enumerate}
    \item \(\langle x, y \rangle = \langle y, x \rangle\) \hfill (Symmetry)
    \item \(\langle cx, y \rangle = c \langle x, y \rangle\) \hfill (Linearity in the first argument)
    \item \(\langle x_1 + x_2, y \rangle = \langle x_1, y \rangle + \langle x_2, y \rangle\) \hfill (Additivity)
    \item \(\langle x, x \rangle \geq 0\), and \(\langle x, x \rangle = 0\) if and only if \( x = 0 \) \hfill (Positivity)
\end{enumerate}
An inner product space is not unique. A vector space may have any amount of inner product spaces as long as the four postulates are met. 

\subsection*{Complex Vector Spaces}
For complex vector spaces, the first postulate changes to:
\[
\langle x, y \rangle = \overline{\langle y, x \rangle},
\]
ensuring that the inner product remains real-valued when evaluated. The other postulates remain similar, with linearity applying only in the first argument to maintain consistency.

\subsection*{Examples}
\begin{itemize}
    \item For \(\mathbb{R}^2\) or \(\mathbb{R}^3\), an example of an inner product is:
    \[
    \langle x, y \rangle = \|x\| \|y\| \cos(\theta),
    \]
    where \(\theta\) is the angle between vectors \( x \) and \( y \). Another example is:
    \[
    \langle x, y \rangle = x_1 y_1 + x_2 y_2 + x_3 y_3.
    \]
    The dot product is an inner product in \(\mathbb{R}^n\).

    \item For integrable functions \( f(x) \) and \( g(x) \) over an interval \([a, b]\), the inner product is defined as:
    \[
    \langle f, g \rangle = \int_a^b f(x) g(x) \, dx.
    \]
\end{itemize}

\subsection*{Integral Example}
Consider \( f(x) = e^{3x}( \cos(3x) + i \sin(3x) ) \) and \( g(x) = e^{-x} ( 2\cos(5x) - i \sin(5x) ) \). The inner product over a suitable interval is:
\[
\langle f, g \rangle = \int_a^b f(x) g(x) \, dx.
\]

The norm squared:
\[
\langle f, f \rangle = \int_a^b f(x)^2 \, dx,
\]
is always positive, and zero only when \( f(x) = 0 \), satisfying the fourth postulate.

\end{document}
